\section{Conclusions}
\label{sec:conclusions}

This work documents the study of different heuristics and metaheuristics, 
essential pieces of problem solving toolboxes, applied in many areas of 
science and engineering~\cite{6603883,Kytojoki2007,becker2013evaluation}. Here 
we focus on a particular case study -- the Capacitated Vehicle Routing Problem 
(CVRP) -- and provide custom implementations of several 
heuristics and metaheuristics, specially oriented to tackle it. We 
describe the implementation of both parallel and 
sequential versions of the Clarke and Wright Savings (CWS) algorithm, several 
local search heuristics (2-opt and 1-interchange), as well as custom versions 
of the Simulated Annealing (SA) algorithm and Genetic Algorithm (GA) as 
examples of metaheuristics. The C\slash C++ code used in the implementation is 
available online\,\footnote{Available in \url{http://paginas.fe.up.pt/~up200400437/cvrp.zip}} for further inspection and was tested against several 
datasets, available in~\cite{website:cvrp-datasets}.\vertbreak

None of the implementations reached the best values stated in the datasets. The 
CWS algorithm proved a strong constructive heuristic, particularly when combined 
with a 2-opt local search procedure, due to the latter's capacity in removing 
crossings between links in `tangled' routes. The results confirm the dominance 
of the parallel version over the sequential one.\vertbreak 

The SA method implemented here clearly benefits from a 1-interchange local 
search procedure, always producing enhancements to the solutions provided by the 
CWS algorithm. Nevertheless, the 
use of initial solutions generated by the CWS algorithm and the use of the same 
1-interchange neighbor generation mechanism seems to force our SA procedure to 
be stuck in local optima, an insight which seems to be supported by the 
fact that similar final solutions are reached even when applying different starting 
parameters.\vertbreak

Similarly to the SA's case, our GA implementation seems to be rather poor as it 
does not provide results as competitive as those generated by the CWS and SA 
algorithms, both when the initial population is entirely composed by randomly 
generated solutions or when it includes solutions generated by the CWS and SA 
methods.



