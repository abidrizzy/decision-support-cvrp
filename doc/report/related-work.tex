\section{Background}
\label{sec:rel-work}

(...) some argue that ICN is not actually the best strategy to achieve the 
benefits it advertises~\cite{Ghodsi2011b} (...)\\

According to recent research on diverse traffic support on 
ICNs~\cite{Tsilopoulos2011, Khan2012}  one trend is to use 
diverse routing\slash forwarding mechanisms at the Network Layer in order to 
provide efficient dissemination of different classes of traffic.\\ 

(...) Tsilopoulos et al.~\cite{Tsilopoulos2011} suggest the 
idea of `pushing down' flow control, congestion control, reliability and error 
control mechanisms to the Network Layer, instead of relying on their 
implementation at upper layers (Transport, Application). This implies that 
routers should be equipped with some level of `cognition', in particular these 
should be aware of the nature of the traffic being routed/forwarded in order to 
make a correct forwarding decision.\\

In~\cite{Tsilopoulos2011}, the dissemination patterns are 
defined by the Subscriber and\slash or Publisher sides, probably at the 
application level and explicitly stated on the Content Name (e.g. the source 
address `ordered by Bob' field in Channels). I couldn't find anything related 
to this subject in subsequent works from these authors. E.g. in the particular 
case of the extensions proposed by Khan et al. in~\cite{Khan2012}, ICN nodes must 
decide which one from a set of multiple Forward Information Bases (FIBs) would 
be used to forward a particular packet. The authors propose the addition of a 
traffic type field to Interest packets in order to find an appropriate FIB. As 
discussed above, parallel research on IP traffic mechanisms tells us this is not 
always the best idea:

\begin{itemize}
    \item (encryption)
    \item (deep packet inspection may be unfeasible or even prohibited)
    \item (intrinsic security on the Network Layer, as proposed by XIA [6], 
            etc.)
    \item (knowledge on particular traffic types, which at the IP level are 
            irrelevant) (even though the use of multiple FIBs also assumes that 
            some degree of knowledge must be present at the Network Layer - our 
            rationale assumes a FIB for traffic types A and B should be present 
            at the router's memory, along with the default FIB - the use of ML 
            might help the autonomic construction of new FIBs through 
            unsupervised learning)
\end{itemize}

D. Trossen et al.~\cite{Trossen2012} propose (...) rather an architecture itself, 
much like XIA~\cite{Anand2011} or FII~\cite{Ghodsi2011a}. 
(...) rendevouz system, in a publish/subscribe fashion (...) the concept of 
dissemination strategy i.e. (...) to be stored as data structures, in parallel 
with the special rendezvous data structures (called scopes) that manage a given 
type of information or content (...) So in this case, the dissemination strategy 
is closely `binded' to a scope, and this association is known by ICN nodes 
(...). Although this seems to invalidate our proposal (...) within a 
dissemination strategy, as noted by the authors when referring to CCNs as 
proposed by PARC (...) following this lead, we envision the implementation of 
the proposed rendezvous, topology formation and forwarding elements to include 
ML techniques to detect the appropriate forwarding sub-mechanisms according to 
a traffic sub-type (...) e.g. CCN would be a dissemination strategy while the 
internal elements of a CCN node would automatically decide between traffic 
diversity (documents, channels, etc.) (...) so a more flexible mapping between 
dissemination strategies and information scopes (PN: which may provide 
scalability benefits).\\

We believe that routers should be equipped with some level of `cognition', in 
order to be aware of the nature of the traffic being routed\slash forwarded. 
Regarding the aspect of traffic identification, it may seem we are proposing 
to solve things `the hard way', however some of the constraints shown above 
also seem to indicate that identifying traffic by inference (e.g. via ML 
techniques) may be one of the `only' ways.

(...)

Off-line training due to the caching nature of CCNs (packet size only).

(...)

We start by borrowing some techniques from research on IP traffic 
classification, specifically using ML, and adapt it to ICNs.

(...)

On a more pragmatic note, the  ML course I'm attending requires a written 
report about a ML topic not discussed in the classes (35\% of the grade). I 
thought it would be interesting to research on the subject of network traffic 
classification, establishing a ICN parallel to the work already published on IP 
traffic classification, i.e. following the ideas listed above. This way, with 
one main `occupation', I would be able to fulfill 3 practical objectives:

\begin{itemize}

    \item Work on a (arguably) novel topic for my Ph.D. which could result in a 
            publication in the end
    \item Produce material for the ML course
    \item Produce material for the ST (Special Topics) course, ideally a paper 
            with a summary of the work

\end{itemize}

(...)

An evolutionary FIA concept, i.e.

(...)

How does the network identify the traffic type of a particular data exchange? 
How are the different forwarding mechanisms 'uploaded' to the routers? 
How does the network know what mechanisms to apply for a given traffic type? Is 
it decided by the application at the Content Holder? Is this embedded in the 
Content Name? Is this decided at the network level?

(...)

Note that this mechanism could benefit from an explicit indication of the 
traffic type at the packet level, while it could also work with on-the-fly 
traffic classification at each hop.

This way (...), and QoS would be truly embedded in the Network Layer, turning 
the Internet into an autonomic system.

(yes, but what if the link immediately between the device and 'advanced' CCN 
router was the lossy one? How could a reliable Data forwarding mechanism work 
then?)

(...)

